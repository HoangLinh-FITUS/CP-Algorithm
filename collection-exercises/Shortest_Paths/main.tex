\documentclass[12pt]{article}

\usepackage[utf8]{vietnam}
\usepackage{parskip}
\usepackage{geometry}

\geometry{ left=2.54cm, right=2.54cm, top=2.54cm, bottom=2.54cm }

\begin{document}

\textbf{Đến Trường} \\

Hùng đã trở thành học sinh của trường chuyên Nguyễn Trãi. 
Là học sinh mới, Hùng không muốn đi học muộn nên đã chuẩn bị khá kỹ càng. 
Chỉ còn lại một công việc khá gay go là Hùng không biết đi đường nào tới trường là nhanh nhất. 
Thường ngày Hùng không quan tâm tới vấn đề này lắm cho nên bây giờ Hùng không biết phải làm sao cả. 

Bản đồ thành phố là gồm có $N$ nút giao thông và $M$ con đường nối các nút giao thông này. 
Có $2$ loại con đường là đường $1$ chiều và đường $2$ chiều. Độ dài của mỗi con đường là một số nguyên dương. 
Nhà Hùng ở nút giao thông $1$ còn trường Nguyễn Trãi ở nút giao thông $N$. Vì một lộ trình đường đi từ nhà Hùng tới 
trường có thể gặp nhiều yếu tố khác như là gặp nhiều đèn đỏ, đi qua công trường xây dựng, $\cdots$ phải giảm tốc độ cho 
nên Hùng muốn biết là có tất cả bao nhiêu lộ trình ngắn nhất đi từ nhà tới trường. 

Bạn hãy lập trình giúp Hùng giải quyết bài toán khó này.

\textbf{Dữ liệu:} 
\begin{itemize}
    \item Dòng thứ nhất ghi hai số nguyên $N$ và $M$.
    \item $M$ dòng tiếp theo, mỗi dòng ghi $4$ số nguyên dương $K, U, V, L$. Trong đó:
        \begin{itemize}
            \item $K = 1$ có nghĩa là có đường đi một chiều từ $U$ đến $V$ với độ dài $L$.
            \item $K = 2$ có nghìa là có đường đi hai chiều giữa $U$ và $V$ với độ dài $L$.
        \end{itemize}        
\end{itemize}

\textbf{Kết quả:} 
\begin{itemize}
    \item Ghi hai số là độ dài đường đi ngắn nhấn và số lượng đường đi ngắn nhất. 
    Biết rằng số lượng đường đi ngắn nhất không vượt quá phạm vì \texttt{int64} trong pascal hay \texttt{long long} trong C++.
\end{itemize}

\textbf{Ràng buộc:}
\begin{itemize}
    \item $1 \le n \le 5000$
    \item $1 \le m \le 20000$
    \item $1 \le L \le 32000$
\end{itemize} 

\begin{tabular}{|p{0.5\linewidth}|p{0.5\linewidth}|}
    \hline 
        \multicolumn{1}{|p{0.5\linewidth}|}{\centering \textit{school.inp}}
        & \multicolumn{1}{|p{0.5\linewidth}|}{\centering \textit{school.out}} \\
    \hline
    3 2 \newline
    1 1 2 3 \newline
    2 2 3 1
    & 
    4 1 \\
    \hline 
\end{tabular}

\end{document}
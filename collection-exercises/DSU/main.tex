\documentclass[12pt]{article}

\usepackage[utf8]{vietnam}
\usepackage{parskip}
\usepackage{geometry}

\geometry{ left=2.54cm, right=2.54cm, top=2.54cm, bottom=2.54cm }

\begin{document}

\textbf{Sắp Bò} \\

Trong một nông trại yên bình, có một đàn bò $N$ con vô cùng thông minh, mỗi con đều mang một số hiệu riêng từ $1$ đến $N$. 
Ban đầu, các thành viên của đàn bò này phân tán ngẫu nhiên ở các vị trí khác nhau trong chuồng. 
Tuy nhiên, để tạo ra sự trật tự, đàn bò đã tìm ra một cách sắp xếp độc đáo: chúng sẽ sử dụng hệ thống $M$ lỗ sâu bí ẩn để di chuyển.

Mỗi lỗ sâu, được đánh số từ $1$ đến $M$, là một đường hầm kỳ diệu nối liền hai vị trí bất kỳ trong chuồng. 
Điều thú vị là, các lỗ sâu này có kích thước khác nhau, được đo bằng chiều rộng $w_i$. 
Hai con bò ở hai đầu của một lỗ sâu có thể đổi chỗ cho nhau một cách nhanh chóng.

Mục tiêu của đàn bò là sắp xếp lại đội hình sao cho mỗi con bò đứng đúng vị trí của mình, tức là con bò số $i$ sẽ đứng ở vị trí số $i$. 
Tuy nhiên, để thực hiện điều này, chúng phải lựa chọn những lỗ sâu có kích thước phù hợp để tránh bị "kẹt".

\textbf{Yêu cầu:} Tìm chiều rộng nhỏ nhất của lỗ sâu mà lũ bò phải chui qua.

\textbf{Đảm bảo:} Luôn có cách sắp xếp hợp lệ.

\textbf{Dữ liệu:} 
\begin{itemize}
    \item Dòng đầu tiên chứa hai số nguyên $N$ và $M$.
    \item Dòng thứ hai gồm $N$ số nguyên $p_1, p_2, \dots, p_N$ thể hiện vị trí ban đầu của từng con bò ($p$ là hoán vị của $1$ đến $N$).
    \item $M$ dòng tiếp theo chứa ba số nguyên $a_i, b_i$ và $w_i$ thể hiện thông tin của lỗ sâu thứ $i$.
\end{itemize}

\textbf{Kết quả:} 
\begin{itemize}
    \item Gồm một số nguyên duy nhất: Chiều rộng nhỏ nhất của lỗ sâu mà lũ bò phải chui qua. Nếu không cần dùng lỗ sâu, xuất ra $-1$.
\end{itemize}

\textbf{Ràng buộc:}
\begin{itemize}
    \item $1 \le N,M \le 10^5$
    \item $1 \le a_i,b_i \le N$ ($a_i \neq b_i$)
    \item $1 \le w_i \le 10^9$
\end{itemize} 

\begin{tabular}{|p{0.5\linewidth}|p{0.5\linewidth}|}
    \hline 
        \multicolumn{1}{|p{0.5\linewidth}|}{\centering \textit{wormsort.inp}}
        & \multicolumn{1}{|p{0.5\linewidth}|}{\centering \textit{wormsort.out}} \\
    \hline
    4 4 \newline 
    3 2 1 4\newline
    1 2 9\newline
    1 3 7\newline
    2 3 10\newline
    2 4 3\newline
    & 9\\
    \hline 
\end{tabular}

\end{document}